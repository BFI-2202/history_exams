\documentclass{article}
\usepackage[utf8]{inputenc}

\usepackage[T2A]{fontenc}
\usepackage[utf8]{inputenc}
\usepackage[russian]{babel}

\usepackage{multienum}
\usepackage{hyperref}
\hypersetup{
    colorlinks,
    citecolor=black,
    filecolor=black,
    linkcolor=black,
    urlcolor=blue
}

\usepackage{geometry}
\geometry{
    left=1cm,right=1cm,
    top=2cm,bottom=2cm
}
\usepackage[svgnames]{xcolor}
\usepackage[tikz]{bclogo}

\title{Справочный материал к экзамену по истории}
\author{Лисид Лаконский}
\date{2022–2023 учебный год}

\newtheorem{definition}{Определение}

\begin{document}
\raggedright

\maketitle
\tableofcontents
\pagebreak

\section{Вопрос №1. Образование древнерусского государства}

\begin{bclogo}[logo=\bcinfo, couleurBarre=orange, noborder=true, couleur=white]{Информация о материале}
    Источник справочного материала по данной теме: Киевская Русь // Википедия. [2023]. Дата обновления: 01.06.2023. URL: \url{https://ru.wikipedia.org/?curid=27&oldid=130794937} (дата обращения: 01.06.2023). 
\end{bclogo}

Древнерусское государство \textbf{возникло на торговом пути «из варяг в греки» на землях восточнославянских племён} — ильменских словен, кривичей, полян и \textbf{финно-угорских племён} — чуди, мери и веси, затем охватив земли других восточнославянских племён — древлян, дреговичей, полочан, радимичей, северян и вятичей.

\hfill 

\textbf{Первое надёжно датируемое известие о руси и государстве русов относится к первой трети IX века}: в \textbf{Бертинских анналах под 839 годом} упомянуты послы кагана народа рос, прибывшие сначала в Константинополь, а оттуда ко двору франкского императора Людовика Благочестивого. Император расследовал цель их прибытия и выяснил, что они из народа свеев (шведов, то есть скандинавов).

В «Повести временных лет» под 859 годом сказано, что финно-угорские и северная часть восточнославянских племён находились в даннической зависимости от варягов, а юго-восточная часть восточнославянских племён — от хазар.

\hfill

\textbf{В 860 году} («Повесть временных лет» ошибочно относит его к 866 году) \textbf{русь совершает первый поход на Константинополь}. Греческие источники связывают с ним так называемое первое крещение Руси, после которого на Руси, возможно, возникла епархия и правящая верхушка (возможно, во главе с Аскольдом) приняла христианство.

Если следовать «Повести временных лет», Государству Рюрика предшествовала конфедерация славянских и финно-угорских племён, изгнавшая, а затем призвавшая варягов.

\hfill

\textbf{В 862 году}, согласно «Повести временных лет» (дата приблизительна, поскольку древнейшая хронология летописи представляет собой результат искусственных калькуляций и исторически малодостоверна), \textbf{славянские и финно-угорские племена, прежде изгнавшие варягов, призвали их на княжение}.

\textbf{В том же 862 году} (дата также приблизительна) \textbf{варяги и дружинники Рюрика Аскольд и Дир, направлявшиеся в Константинополь, подчинили себе Киев, тем самым установив полный контроль над важнейшим торговым путём «из варяг в греки»}. При этом Новгородская и Никоновская летописи не связывают Аскольда и Дира с Рюриком, а хроника Яна Длугоша и Густынская летопись называют их потомками Кия.

\textbf{В 879 году в Новгороде умер Рюрик. Княжение было передано Олегу, регенту при малолетнем сыне Рюрика Игоре}. 

\pagebreak
\section{Вопрос №2. Социальный и экономический строй Киевской Руси}

\subsection{Социальный строй Киевской Руси}

\begin{bclogo}[logo=\bcinfo, couleurBarre=orange, noborder=true, couleur=white]{Информация о материале}
    Источник справочного материала по данной теме: Политический и социальный строй Древней Руси // История Отечества. Курс лекций : сайт. – URL: \href{https://siblec.ru/gumanitarnye-nauki/istoriya-otechestva/2-politicheskij-i-sotsialnyj-stroj-drevnej-rusi}{https://siblec.ru/gumanitarnye-nauki/istoriya-otechestva/2-politicheskij-i-sotsialnyj-stroj-drevnej-rusi} (дата обращения: 03.06.2023)
\end{bclogo}

Наверху социальной лестницы стояли \textbf{князь} и \textbf{бояре, местная аристократия из потомков племенных князей и родовых старшин}, \textbf{ближайшее окружение князя} («княжи мужи») и \textbf{дружинники}, со временем все более тяготеющие к земле. В конце XI в. и в начале XII в. они превратились в крупных землевладельцев, владевших боярскими селами, в которых работало в основном несвободное и полусвободное население.

\hfill

Средний класс — это \textbf{городские ремесленники и торговцы, младшая дружина князя}.

\hfill

Низший класс свободного населения — основная масса свободных крестьян, рабочий городской люд, носившие общее название \textbf{«смерды»}. Смерды обладали личной свободой и объединялись в территориальные общины — \textbf{вервь}.

Смерд платил дань князю. Князь распоряжался его землёй.

\textbf{Жизнь смерда была связана с общиной, с круговой порукой}. Например, если вор или убийца не были обнаружены или его укрывали, то виру платила вся община. Вервь могла помочь члену общины уплатить штраф, если он сам участвует в таких общих платежах.

Постепенно \textbf{положение смердов ухудшалось и они оказывались в зависимости от частных землевладельцев или монастырей}, которые становились крупными землевладельцами.

\hfill

В «Русской Правде» упоминается и такая социальная категория, как \textbf{изгои}. Это \textbf{люди, выбитые жизненными условиями из своей социальной группы}: сын попа, не знающий грамоты, промотавшийся купец, отпущенный на волю холоп. Изгои живут в княжеских или церковных селах, полностью зависят от князя и церкви.

\hfill

Низшая ступень социальной пирамиды — \textbf{холопы} (несвободное население). Холоп — это челядин, раб (множественное число — челядь, женщина — роба). Численность холопов росла за счет военнопленных, но главным образом за счет свободных людей. «Русская Правда» перечисляет возможные жизненные ситуации, которые вели к потере личной свободы, превращали человека в обельного (полного) холопа. Это:

\begin{multienumerate}
    \mitemxxx{покупка холопа при свидетелях}{женитьба на робе без ряда, без договора}{неуплата долга}
    \mitemxx{наем в услужение другому без условий (ряда)}{несвобода детей холопов и холопок}
\end{multienumerate}

\textbf{Господин имел неограниченное право распоряжаться жизнью и имуществом холопа, вплоть до его убийства}. Холоп не мог быть свидетелем в суде.

\hfill

Близко к холопству было положение закупов. \textbf{Закуп — это человек, получивший купу (заем) деньгами, землею, инвентарем, семенами и т.д}. До выплаты долга и положенных процентов он \textbf{находился в распоряжении заимодавца на положении холопа}. Несостоятельный, а также провинившийся (порча хозяйского инвентаря, рабочего скота, недосмотр за хозяйским добром) должник превращался в полного холопа. Известна категория \textbf{«закупа ролейного»}, который вспахивал хозяйскую пашню, смотрел за скотом, выполнял другие сельскохозяйственные работы, пользуясь инвентарем и скотом хозяина. Закуп мог работать, пользуясь хозяйским инвентарем и скотом на своем поле.

\textbf{Закуп находился под защитой публичного права}. Он мог искать судебной защиты, жаловаться на господина, не мог быть продан в холопы. Если не было других свидетелей, он мог быть свидетелем в суде. Штраф за побои для закупа назначался такой же, как для свободного. Но господин имел право его побить «за дело». Всякая отлучка считалась бегством и сурово наказывалась. Бежавший закуп превращался в холопа. \textbf{Холопы и закупы составляли главную рабочую силу в обширных хозяйствах господствующего класса}.

\hfill

Правовые документы Древней Руси регулировали проблему наследства. \textbf{Семейным имуществом распоряжался мужчина, глава семьи}. \textbf{Право наследования имели сыновья}. Жена в случае смерти мужа получала свое приданое. Отец и братья обязаны были выдать своих дочерей и сестер замуж, обеспечив их приданым.

\subsection{Экономический строй Киевской Руси}

\begin{bclogo}[logo=\bcinfo, couleurBarre=orange, noborder=true, couleur=white]{Информация о материале}
    Источник справочного материала по данной теме: репозиторий \href{https://github.com/BFI-2202/history_notes}{BFI-2202/history\_notes}, автор репозитория — \textbf{Лисид Лаконский}
\end{bclogo}

Для экономики Древнерусского государства периода 9-12 веков характерен \textbf{ранний феодализм}.
Зарождалась сама основа взаимоотношений между государством, сельским хозяйством и феодалами. \textbf{Ядром русской экономики в то время считалось именно сельское хозяйство, которое занимало главенствующее положение}.

Данный период характеризуется \textbf{развитостью товарного хозяйства}, поскольку производилось почти все необходимое. \textbf{Ремесленное дело} развивалось быстрыми темпами, а \textbf{центрами становились города}, между тем и в селах развивались отдельные отрасли. Важнейшую роль играла \textbf{черная металлургия}, поскольку в Древней Руси имелись богатые болотные руды. Всевозможными способами обрабатывалось железо, из него изготавливались различные изделия для хозяйственных нужд, военного дела и так далее.

\hfill

Торговля в Древней Руси играла огромное значение, особенно внешняя. \textbf{Внешняя торговля была довольно сильно развита, являлась важной составляющей экономики древнерусских княжеств}. Из Руси \textbf{вывозились на продажу пушнина, воск, мёд, смола, лён и льняные ткани, серебряные вещи, пряслица из розового шифера, оружие, замки, резная кость и прочее}. А предметами ввоза были предметы роскоши, фрукты, пряности, краски и прочее.

\hfill

Формой налогов в Древней Руси выступала \textbf{дань}, которую выплачивали подвластные племена. Чаще всего \textbf{единицей налогообложения выступал «дым», то есть дом, или семейный очаг}. Размер налога традиционно был в \textbf{одну шкурку с дыма}. В некоторых случаях — например, с племени вятичей, — бралось по монете от рала (плуга). \textbf{Формой сбора дани было полюдье}, когда князь с дружиной с ноября по апрель объезжал подданных.

В \textbf{946 году} после подавления восстания древлян княгиня Ольга провела налоговую реформу, \textbf{упорядочив сбор дани}. Она \textbf{отменила полюдье и установила «уроки», то есть размеры дани}, и создала «погосты» — крепости на пути полюдья, в которых жили княжеские управляющие и куда свозилась дань. \textbf{Такая форма сбора дани и сама дань назывались «повоз»}. При уплате налога подданные получали глиняные печати с княжеским знаком, что освобождало их от повторного сбора. Реформа \textbf{содействовала централизации великокняжеской власти и ослаблению власти племенных князей}.

\pagebreak
\section{Вопрос №3. Политический и сословный строй Киевского государства}

\subsection{Политической строй Киевского государства}

\begin{bclogo}[logo=\bcinfo, couleurBarre=orange, noborder=true, couleur=white]{Информация о материале}
    Источник справочного материала по данной теме: Политический и социальный строй Древней Руси // История Отечества. Курс лекций : сайт. – URL: \href{https://siblec.ru/gumanitarnye-nauki/istoriya-otechestva/2-politicheskij-i-sotsialnyj-stroj-drevnej-rusi}{https://siblec.ru/gumanitarnye-nauki/istoriya-otechestva/2-politicheskij-i-sotsialnyj-stroj-drevnej-rusi} (дата обращения: 03.06.2023)
\end{bclogo}

В процессе образования государственности сформировалась и упрочилась власть \textbf{князя}, верховного правителя племенных образований, а затем и всего Древнерусского государства. \textbf{Князь был высшим судьей и руководителем местной власти}. Судебные штрафы шли в княжескую казну.

Князь \textbf{являлся высшим военачальником}. Он организовал оборону страны, сражался во главе своей дружины, заключал мирные договоры. \textbf{Осуществляя внешнюю политику государства, он являлся главным организатором внешней торговли}, которая стала важнейшей отраслью экономики Древней Руси: князь заключал торговые договоры с Византией и заботился об их выполнении, применяя, если нужно, вооруженную силу. Княжеская дружина охраняла товары на торговых путях.

\hfill

Неотъемлемой частью княжеской власти была его \textbf{дружина}, с которой он был неразлучен. Это его ближайшие сподвижники в ратных подвигах, товарищи и советники. Дружина была, как правило, сравнительно невелика и состояла из 700—800 храбрых, обученных и преданных князю воинов, которых он сам подбирал.

\textbf{Князь был обязан постоянно советоваться со своей дружиной, прислушиваться к ее мнению}. Святослав, например, отказался принять христианство только потому, что дружина была против. Игорь погиб из-за того, что, послушавшись совета своей дружины, пошел вторично собирать дань. Владимир Святославович победил своего брата Ярополка благодаря поддержке его Блудом, дружинником Ярополка.

Князь заботился о своей дружине, раздавая ей серебро и злато, лучшую пищу и питье. Дружина делила с князем лавры победы и горечь поражения. После битвы победивший князь расправлялся с дружинниками своего побежденного противника.

\hfill

Княжеская \textbf{дружина не была однородна}. Уже в Х в. она \textbf{делилась на старшую дружину — бояре, или боляре} (от слова “большие”), и \textbf{младшую — молодь, отроки, гридь}. Ярослав, кроме того, упоминает \textbf{нарочитых, или добрых, людей, не входящих в дружину}. Иногда князь собирал ополчение из горожан и свободного сельского населения, которое участвовало в битвах. Так, новгородское ополчение помогло Ярославу Мудрому в борьбе с его братом Святополком за Киевский стол. Постепенно бояре-дружинники перестают жить с князем “на един хлебе”, а \textbf{оседают на земле, владеют селами, превращаются в землевладельцев-феодалов}.

\hfill

\textbf{Издревле среди славянских племен существовало вече — народное собрание}. С образованием государства и усилением власти князя \textbf{роль веча уменьшалась}. В более полном виде оно сохранилось в Новгороде и Пскове. Но и в других княжествах власть князя не была беспредельной: он \textbf{должен был считаться с волеизъявлением народа}. Население Киева и других крупных городов, например, вольно было принять или не принять князя. Вече решало вопросы войны и мира, и прежде всего тогда, когда дружине князя нужно было выставить в помощь ополчение. Иногда народ сам выступал инициатором войны, а в других случаях требовал от князя заключения мира. В междоусобных войнах вече чаще всего выступало как инициатор примирения, но не всегда. Известны случаи, когда жители Киева требовали от князя убрать неугодного им тиуна (управляющего делами князя), который притеснял их поборами и самоуправством. Князь вынужден был выполнить это требование и даже обещал впредь советоваться с ними при назначении тиуна.

\hfill

Как видим, полномочия веча были довольно широки, никакой закон не ограничивал и не определял их. Собрания начинались стихийно, чаще всего при возникновении какой-либо острой ситуации. Иногда ополчение собирало вече во время похода. \textbf{Нельзя также преувеличивать роль веча}. Когда власть князя была крепка, он действовал самостоятельно, и совет веча был для него необязателен. Но \textbf{во многих случаях он должен был считаться с волей народа}.

\subsection{Сословный строй Киевского государства}

\begin{bclogo}[logo=\bcinfo, couleurBarre=orange, noborder=true, couleur=white]{Примечание}
    Данный материал доступен в справочном материале по предыдущему вопросу
\end{bclogo}

\pagebreak
\section{Вопрос №4. «Русская Правда» — основной свод законов древнерусского государства}

Русская Правда — \textbf{сборник правовых норм} Киевской Руси, датированный различными годами, начиная с 1016 года, \textbf{древнейший русский правовой кодекс}. Является одним из основных письменных источников русского права. Происхождение наиболее ранней части Русской Правды связано с деятельностью князя \textbf{Ярослава Мудрого}. Написана на древнерусском языке. Русская Правда стала основой русского законодательства и сохраняла своё значение до XV—XVI веков.

\subsection{Право по Русской Правде}

\subsubsection{Уголовное право}

Как и другие ранние правовые памятники, Русская Правда \textbf{отличает убийство неумышленное}, «в сваде», то есть во время ссоры, \textbf{от умышленного} — «в обиду», и от убийства «в разбое». \textbf{Различалось причинение тяжкого или слабого ущерба, а также действия, наиболее оскорбительные для пострадавшего}

\textbf{Уголовные санкции}

\begin{enumerate}
    \item Правда Ярослава санкционировала \textbf{кровную месть}, но ограничивала круг мстителей определёнными ближайшими родственниками убитого
    \item Штрафы в пользу князя:
    \begin{enumerate}
        \item \textbf{Вира} — штраф за убийство свободного человека
        \item \textbf{Полувирье} — штраф за тяжкие увечья свободному человеку
        \item \textbf{Продажа} — штраф за другие уголовные преступления — нанесение менее тяжких телесных повреждений, кражу и другое
    \end{enumerate}
    \item Плата пострадавшим:
    \begin{enumerate}
        \item \textbf{Головничество} — плата в пользу родственников убитого
        \item \textbf{Плата «за обиду»} — как правило, плата потерпевшему
        \item \textbf{Урок} — плата хозяину за украденную или испорченную вещь или за убитого холопа
    \end{enumerate}
    \item Наиболее тяжкими преступлениями считались \textbf{разбой «безъ всякоя свады», поджог гумна или двора и конокрадство}. За них преступник подвергался потоку и разграблению. Первоначально это была высылка преступника и конфискация имущества, позднее — \textbf{преступник и его семья обращались в рабство, а его имущество подвергалось разграблению}. Поток и разграбление инициировала община, а осуществляла княжеская власть, то есть \textbf{эта мера наказания уже была поставлена под контроль государства}
\end{enumerate}

\subsubsection{Частное право}

По Русской Правде купец мог \textbf{отдавать имущество на хранение} (поклажа). Совершались \textbf{ростовщические операции}: в рост давались деньги — отданное (исто) возвращалось с процентами (резы), или продукты с возвратом в пропорционально большем размере. Подробно представлены нормы наследственного права. \textbf{Предусматривалось наследование как по закону, так и по завещанию}

\subsubsection{Процессуальное право}

\textbf{Уголовные правонарушения рассматривал княжий (княжеский) суд} — суд, осуществлявшийся представителем князя. Пойманного на дворе в ночное время вора можно было убить на месте или вести на княжий суд.

По гражданским делам процесс носил состязательный (обвинительный) характер, при котором \textbf{стороны были равноправными и сами осуществляли процессуальные действия}.

\subsection{Социальные категории по Русской Правде}

\textbf{Знать и привилегированные слуги}

\begin{enumerate}
    \item \textbf{Знать в Русской Правде представлена князем и его старшими дружинниками — боярами}. Князю идут штрафы, имущество которого защищают некоторые статьи, и именем которого вершится суд.
    \item \textbf{Привилегированное положение имели тиуны, огнищане — высокопоставленные княжеские и боярские слуги}, а также княжеский старший конюх
\end{enumerate}

\hfill

\textbf{Рядовые свободные жители}

\begin{enumerate}
    \item Основное действующее лицо Русской Правды — \textbf{муж — свободный мужчина}
    \item Русин — житель Киевской Руси; \textbf{дружинник: гридин — представитель боевой дружины}
    \item \textbf{Купчина} — дружинник, занимавшийся торговлей
    \item \textbf{Ябетник} — дружинник, связанный с судебным процессом
    \item \textbf{Мечник} — сборщик штрафов
    \item \textbf{Изгой} — человек, потерявший связь с общиной
\end{enumerate}

\hfill

\textbf{Зависимое население.} Привилегированное положение среди зависимых людей имели \textbf{княжеские кормильцы, а также княжеские сельские и ратайные старосты}

Низшее положение занимали \textbf{смерды, холопы, рядовичи и закупы}

\begin{enumerate}
    \item \textbf{Смерд} — крестьянин, в этом контексте зависимый крестьянин
    \item \textbf{Холопство} могло быть \textbf{обельным} (полным) или \textbf{закупным}. Обель — пожизненный раб.
    \item \textbf{Закуп} — свободный человек, взявший купу — кредит, и попавший в зависимость до тех пор, пока не отдаст или не отработает этот долг
    \item \textbf{Рядович} — лицо, поступившее на службу и ставшее зависимым по «ряду», то есть договору
\end{enumerate}

\pagebreak
\section{Вопрос №5. Введение христианства на Руси: причины и историческое значение}

\textbf{Крещение} \textbf{Руси} — термин, под которым в современной исторической науке понимается комплекс взаимосвязанных событий в основном конца X века, включавших личное крещение киевского князя \textbf{Владимира Святославича}, его окружения и населения крупнейших городов Киевской Руси, а также осуществленные княжеской властью меры по созданию на Руси церковной организации и среды; рассматривается как принятие христианства в качестве официальной религии Киевской Руси. Исторические источники дают противоречивые указания на точное время принятия новой религии. Традиционно, вслед за летописной хронологией, событие принято относить к \textbf{988 году} и считать началом официальной истории Русской церкви.

\subsection{Причины принятия христианства на Руси}

\begin{enumerate}
    \item Необходимость объединения племен на новой духовной основе
    \item Необходимость укрепления власти Киевского князя 
    \item Стремление укрепить международный авторитет Киевской Руси.
    \item Необходимость приобщения Руси к общеевропейским духовным и культурным ценностям
    \item Личные мотивы князя Владимира
\end{enumerate}

\subsection{Значение принятия христианства на Руси}

\begin{enumerate}
    \item Сплочение народа вокруг единой религии
    \item Улучшение международного положения страны, за счет принятия религии, которая существовала в странах-соседях.
    \item Развитие христианской культуры, которая пришла в страну вместе с религией.
    \item Укрепление власти князя в стране
\end{enumerate}

\pagebreak
\section{Вопрос №6. Причины и хронологические рамки периода феодальной раздробленности}

\subsection{Причины раздробленности}

\begin{enumerate}
    \item \textbf{Лестничная (родовая) система наследования престола}. Эта система с одной стороны постоянно увеличивала количество наследников, а с другой стороны также увеличивала число князей-изгоев. Все это вело к междоусобным войнами и ситуациям, когда князья делили страну между собой.
    \item \textbf{Развитие земледелия}. Благодаря этому процессу многие дружинники стали землевладельцами. С развитием этого процесса землевладельцы становились крупнее и финансово сильнее. Крупных земледельцев очень не устраивала лестничная система наследования престола. Они всячески пытались ограничить власть князя или добиться завершения княжеских переходов.
    \item \textbf{Развитие ремесла}. развитие ремесла имело важное следствие - рост городов и превращение их в культурные и политические центры.
    \item \textbf{Падение торгового пути «из варяг в греки»}. Этот путь проходил через Русь и был тем экономическим стимулом, который заставлял страну держаться единой.
\end{enumerate}

\subsection{Хронологические рамки периода феодальной раздробленности}

В современной отечественной исторической науке можно выделить 5 подходов к ответу на вопрос о начале феодальной раздробленности:

\begin{enumerate}
    \item \textbf{1054 год} (\textbf{смерть Ярослава Мудрого}). Такие историки как Карамзин, Насонов утверждали, что именно смерть Ярослава Мудрого является той чертой, после которой началась раздробленность.
    \item \textbf{1097 год} (\textbf{Любечский съезд князей}). Этой версии придерживаются историки Лихачев, Греков. Они утверждают, что именно на Любечском съезде был утвержден феодальный принцип, что каждый сам «держит» свою землю.
    \item \textbf{1132 год} (\textbf{смерть Мстислава Великого}). Это версия историков Сахарова, Рыбакова, Кузьмина. Они утверждают, что распад Руси на отдельный княжества стал возможным только после смерти князя Мстислава.
    \item \textbf{1243 год} (\textbf{начало татаро-монгольского ига}). Эту версию озвучивают историки Кожников и Бегунов. Они утверждают, что распад был следствием вторжения монголов.
    \item В современной науке все чаще публикуются мнения авторов, что никакого единого древнерусского государства не существовало, и что \textbf{феодальная раздробленность это начальная стадия становления нашего государства}. Это версия таких историков как Дьяконов, Дворниченко и других.
\end{enumerate}

После \textbf{вокняжения Ивана III} (\textbf{1462}) процесс объединения русских княжеств под властью Москвы вступил в решающую фазу.

\textbf{В 1478 году был присоединён Новгород}, \textbf{в 1480 году была достигнута независимость от Орды}, \textbf{в 1487 году начался процесс перехода подвластных Литве князей с их землями под власть Москвы}.

\hfill

К концу правления Василия III (\textbf{1533}) \textbf{Москва стала центром Русского централизованного государства}, присоединив помимо всей Северо-Восточной Руси и Новгорода также смоленские и черниговские земли, отвоёванные у Литвы.

\hfill

\textbf{16 января 1547 года} великий князь Московский \textbf{Иван IV был венчан на царство}. Он распустил все уделы и объявил себя единственным царем.

\pagebreak
\section{Вопрос №7. Монголо-татарское нашествие и его последствия}

\subsection{Начало нашествия и предпосылки}

Впервые войска Руси и Орды сошлись \textbf{31 мая 1223 году} в \textbf{сражении на Калке}. Русские войска вел киевский князь \textbf{Мстислав}, а противостояли им \textbf{Субедей} и \textbf{Джубе}. Русское войско было не просто повержено, оно было фактически уничтожено.

\subsection{Нашествие 1237–1238 годов}

В \textbf{1236 году} монголы начали очередной поход против половцев. В этом походе они добились большого успеха и во второй половине \textbf{1237 года} подошли к границам рязанского княжества.  Командовал азиатской конницей хан \textbf{Батый}, внук Чингисхана. В его подчинении было \textbf{150 тысяч человек}. С ним в походе участвовал \textbf{Субедей}, который был знаком с русичами по предыдущим столкновениям.

\hfill

Вторжение произошло в \textbf{начале зимы 1237 года} (по другому мнению — \textbf{поздней осенью}). С огромной скоростью конница монголов передвигалась по стране, покоряя один город за другим:

\begin{enumerate}
    \item \textbf{Рязань} – пала в конце декабря 1237 года. Осада длилась 6 дней.
    \item \textbf{Москва} – пала в январе 1238 года. Осада длилась 4 дня. Этому событию предшествовала битва под Коломной, где Юрий Всеволодович со своим войском пытался остановить врага, но был разбит.
    \item \textbf{Владимир} – пал в феврале 1238 года. Осада длилась 8 дней. 
\end{enumerate}

После взятия Владимира \textbf{фактически все восточные и северные земли} оказались в руках Батыя. Он покорял один город за другим (Тверь, Юрьев, Суздаль, Переславль, Дмитров). В начале марта пал \textbf{Торжок}, открыв тем самым путь монгольскому войску на север, к Новгороду. Но Батый совершил другой маневр и вместо похода на Новгород, он развернул свои войска и отправился штурмовать \textbf{Козельск} — и, понеся большие потери, развернулся в степи. Так завершился первый поход и первое нашествие татаро-монгольского войска на Русь.

\subsection{Нашествие 1239–1242 годов}

После перерыва в полтора года, в 1239 году началось новое нашествие на Русь войск хана Батыя. В этом году основные события происходили в \textbf{Переяславле и Чернигове}. Вялость наступления Батыя связана с тем, что в это время он вел активную борьбу с половцами, в частности на территории Крыма.

\hfill

\textbf{Осенью 1240} года Батый привел свое войско под стены \textbf{Киева}. Древняя столица Руси не смогла долго сопротивляться. Город пал 6 декабря 1240 года. Историки отмечают особое зверство, с которым вели себя захватчики. Киев был практически полностью уничтожен. После этих событий армия захватчиков разделилась:

\begin{enumerate}
    \item Часть отправилась на \textbf{Владимир-Волынский}.
    \item Часть отправилась на \textbf{Галич}.
\end{enumerate}

\subsection{Последствия татаро-монгольского нашествия на Русь}

Последствия нашествия азиатского войска на Русь историки описывают однозначно:

\begin{enumerate}
    \item \textbf{Страна была покорена, и стала полностью зависимой от Золотой Орды}.
    \item \textbf{Русь начала ежегодно платить дань победителям} (деньгами и людьми).
    \item \textbf{Страна впала в ступор в плане прогресса и развития из-за непосильного ига}.
\end{enumerate}

\pagebreak
\section{Вопрос №8. Москва — центр объединения русских земель}

Возвышение Москвы — политический процесс, который проходил в \textbf{14-15 веках}. Этот процесс интересен тем, что \textbf{в результате борьбы между удельными княжествами победу одержала Москва} — город, который буквально за 100 лет до этого был провинциальным, а в момент возвышения не отличался ни богатством, ни условиями. Но благодаря симбиозу Москвы, Орды и церкви этот процесс стал возможным.

\subsection{Этапы объединения русских земель вокруг Москвы}

Весь процесс возвышение Московского княжества и объединения русских земель можно свести к 3-м основным этапам:

\begin{enumerate}
    \item Конец 13 века — 80-е годы 14 века
    \item 90-е годы 14 века — 1462 год
    \item 1462 — 1533 годы
\end{enumerate}

\subsubsection{Первый этап — до 80-х годов 14 века}

Этот этап \textbf{характеризуется борьбой между Москвой, Тверью и частично Литвой}. Главный процесс, который происходил в это время — борьба отдельных княжеств северо-восточной Руси за главенствующее положение. В результате \textbf{Москве удалось закрепить ярлык на Великое княжение за собой}. Важный момент — московский князь получил ярлык на Великое княжение, но назывался он князем владимирским.

\subsubsection{Второй этап (конец 14 века – 1462)}

На втором этапе \textbf{борьба продолжалась между Москвой, Тверью и Литвой}. Именно на этом этапе у Литовского княжества появилась возможность захвата не только Москвы, но и остальных удельных княжеств Руси. Это стало возможным из-за \textbf{ослабления позиций Орды}, которая в 1395 году потерпела крупное поражение от Тамерлана. Однако в \textbf{1399 году она потерпела поражение от Орды, что создало хорошие условия для Москвы}. Тверь на этом этапе начала отходить на второстепенный план, и к началу третьего этапа возвышения Москвы Тверь утратила свой статус, подчинившись некогда более скромному соседу.

\subsubsection{Третий этап (1462–1533)}

На третьем этапе возвышения Москвы основная борьба шла между Москвой и Великим княжеством Литовским. Остальные удельные княжества Руси уже практически полностью признали власть Москвы, и оставался только один альтернативный путь развития — Литовское княжество. В конечном итоге \textbf{победу Москвы предопределили политические системы двух центров}. \textbf{Москва пошла по пути централизации власти, когда основное управление строилась вокруг князя}. \textbf{В великом княжестве литовском фактически была олигархия}. В результате это \textbf{привело к нагноению системы управления и дальнейшего позитивного политического развития у данного княжества не было}.

\pagebreak
\section{Вопрос №9. Русь между Ордой и Литвой в 14–15 веках}

К \textbf{середине XIII в.} русские земли оказались между Золотой Ордой и Великим княжеством Литовским. В Прибалтике на землях, населенных литовскими племенами возникло раннефеодальное государство. Его основателем считают \textbf{князя Миндовга}. Русские летописи впервые упоминают о нем в \textbf{1219 г.} В состав Литовского государства с момента его возникновения входили \textbf{земли в бассейне реки Неман} (города Новогрудок, Гродно и др.), так называемая \textbf{Черная Русь}.

\hfill

\textbf{Галицкое княжество} вошло в состав Польши, \textbf{земли южной и юго-западной Руси} (Киев, Волынь, Подолье и др.) после завоевания монголами платили дань Орде. Однако в связи с усилением Литовского государства после битвы при Синей Воле (приток Южного Буга) с Ордой (1363), эти земли \textbf{вошли в состав Великого княжества Литовского и Русского}.

\hfill

Центр русской политической жизни переместился в \textbf{северо-восточную} (Владимиро-Суздальскую) и \textbf{северо-западную} (Новгородскую) Русь.

\hfill

Апогей раздробленности северо-восточной Руси приходится на рубеж XIII—XIV вв. Тогда \textbf{на землях Владимиро-Суздальского княжества сложились 14 удельных княжеств} (Суздальское, Ростовское, Ярославское, Тверское, Московское, Переяславское и др.), в свою очередь, делившихся на еще более мелкие владения. Главой северо-восточной Руси правители Золотой Орды считали \textbf{великого князя Владимирского}. Им должен был становиться старший в роду из потомков Всеволода Большое Гнездо. Однако удельные князья вскоре \textbf{нарушили этот порядок, вступив в борьбу за великое княжение Владимирское, исходя из могущества своих княжеств и расположенности к ним ордынских ханов}. В этой борьбе за главенство среди русских земель наиболее активно действовали \textbf{тверские и московские князья}.

\end{document}