\documentclass{article}
\usepackage[utf8]{inputenc}

\usepackage[T2A]{fontenc}
\usepackage[utf8]{inputenc}
\usepackage[russian]{babel}

\usepackage{multienum}
\usepackage{hyperref}
\hypersetup{
    colorlinks,
    citecolor=black,
    filecolor=black,
    linkcolor=black,
    urlcolor=blue
}

\usepackage{geometry}
\geometry{
    left=1cm,right=1cm,
    top=2cm,bottom=2cm
}
\usepackage[svgnames]{xcolor}
\usepackage[tikz]{bclogo}

\title{Справочный материал к экзамену по истории}
\author{Лисид Лаконский}
\date{2022–2023 учебный год}

\newtheorem{definition}{Определение}

\begin{document}
\raggedright

\maketitle
\tableofcontents
\pagebreak

\section{Вопрос №1. Образование древнерусского государства}

\begin{bclogo}[logo=\bcinfo, couleurBarre=orange, noborder=true, couleur=white]{Информация о материале}
    Источник справочного материала по данной теме: Киевская Русь // Википедия. [2023]. Дата обновления: 01.06.2023. URL: \url{https://ru.wikipedia.org/?curid=27&oldid=130794937} (дата обращения: 01.06.2023). 
\end{bclogo}

Древнерусское государство \textbf{возникло на торговом пути «из варяг в греки» на землях восточнославянских племён} — ильменских словен, кривичей, полян и \textbf{финно-угорских племён} — чуди, мери и веси, затем охватив земли других восточнославянских племён — древлян, дреговичей, полочан, радимичей, северян и вятичей.

\hfill 

\textbf{Первое надёжно датируемое известие о руси и государстве русов относится к первой трети IX века}: в \textbf{Бертинских анналах под 839 годом} упомянуты послы кагана народа рос, прибывшие сначала в Константинополь, а оттуда ко двору франкского императора Людовика Благочестивого. Император расследовал цель их прибытия и выяснил, что они из народа свеев (шведов, то есть скандинавов).

В «Повести временных лет» под 859 годом сказано, что финно-угорские и северная часть восточнославянских племён находились в даннической зависимости от варягов, а юго-восточная часть восточнославянских племён — от хазар.

\hfill

\textbf{В 860 году} («Повесть временных лет» ошибочно относит его к 866 году) \textbf{русь совершает первый поход на Константинополь}. Греческие источники связывают с ним так называемое первое крещение Руси, после которого на Руси, возможно, возникла епархия и правящая верхушка (возможно, во главе с Аскольдом) приняла христианство.

Если следовать «Повести временных лет», Государству Рюрика предшествовала конфедерация славянских и финно-угорских племён, изгнавшая, а затем призвавшая варягов.

\hfill

\textbf{В 862 году}, согласно «Повести временных лет» (дата приблизительна, поскольку древнейшая хронология летописи представляет собой результат искусственных калькуляций и исторически малодостоверна), \textbf{славянские и финно-угорские племена, прежде изгнавшие варягов, призвали их на княжение}.

\textbf{В том же 862 году} (дата также приблизительна) \textbf{варяги и дружинники Рюрика Аскольд и Дир, направлявшиеся в Константинополь, подчинили себе Киев, тем самым установив полный контроль над важнейшим торговым путём «из варяг в греки»}. При этом Новгородская и Никоновская летописи не связывают Аскольда и Дира с Рюриком, а хроника Яна Длугоша и Густынская летопись называют их потомками Кия.

\textbf{В 879 году в Новгороде умер Рюрик. Княжение было передано Олегу, регенту при малолетнем сыне Рюрика Игоре}. 

\pagebreak
\section{Вопрос №2. Социальный и экономический строй Киевской Руси}

\subsection{Социальный строй Киевской Руси}

\begin{bclogo}[logo=\bcinfo, couleurBarre=orange, noborder=true, couleur=white]{Информация о материале}
    Источник справочного материала по данной теме: Политический и социальный строй Древней Руси // История Отечества. Курс лекций : сайт. – URL: \href{https://siblec.ru/gumanitarnye-nauki/istoriya-otechestva/2-politicheskij-i-sotsialnyj-stroj-drevnej-rusi}{https://siblec.ru/gumanitarnye-nauki/istoriya-otechestva/2-politicheskij-i-sotsialnyj-stroj-drevnej-rusi} (дата обращения: 03.06.2023)
\end{bclogo}

Наверху социальной лестницы стояли \textbf{князь} и \textbf{бояре, местная аристократия из потомков племенных князей и родовых старшин}, \textbf{ближайшее окружение князя} («княжи мужи») и \textbf{дружинники}, со временем все более тяготеющие к земле. В конце XI в. и в начале XII в. они превратились в крупных землевладельцев, владевших боярскими селами, в которых работало в основном несвободное и полусвободное население.

\hfill

Средний класс — это \textbf{городские ремесленники и торговцы, младшая дружина князя}.

\hfill

Низший класс свободного населения — основная масса свободных крестьян, рабочий городской люд, носившие общее название \textbf{«смерды»}. Смерды обладали личной свободой и объединялись в территориальные общины — \textbf{вервь}.

Смерд платил дань князю. Князь распоряжался его землёй.

\textbf{Жизнь смерда была связана с общиной, с круговой порукой}. Например, если вор или убийца не были обнаружены или его укрывали, то виру платила вся община. Вервь могла помочь члену общины уплатить штраф, если он сам участвует в таких общих платежах.

Постепенно \textbf{положение смердов ухудшалось и они оказывались в зависимости от частных землевладельцев или монастырей}, которые становились крупными землевладельцами.

\hfill

В «Русской Правде» упоминается и такая социальная категория, как \textbf{изгои}. Это \textbf{люди, выбитые жизненными условиями из своей социальной группы}: сын попа, не знающий грамоты, промотавшийся купец, отпущенный на волю холоп. Изгои живут в княжеских или церковных селах, полностью зависят от князя и церкви.

\hfill

Низшая ступень социальной пирамиды — \textbf{холопы} (несвободное население). Холоп — это челядин, раб (множественное число — челядь, женщина — роба). Численность холопов росла за счет военнопленных, но главным образом за счет свободных людей. «Русская Правда» перечисляет возможные жизненные ситуации, которые вели к потере личной свободы, превращали человека в обельного (полного) холопа. Это:

\begin{multienumerate}
    \mitemxxx{покупка холопа при свидетелях}{женитьба на робе без ряда, без договора}{неуплата долга}
    \mitemxx{наем в услужение другому без условий (ряда)}{несвобода детей холопов и холопок}
\end{multienumerate}

\textbf{Господин имел неограниченное право распоряжаться жизнью и имуществом холопа, вплоть до его убийства}. Холоп не мог быть свидетелем в суде.

\hfill

Близко к холопству было положение закупов. \textbf{Закуп — это человек, получивший купу (заем) деньгами, землею, инвентарем, семенами и т.д}. До выплаты долга и положенных процентов он \textbf{находился в распоряжении заимодавца на положении холопа}. Несостоятельный, а также провинившийся (порча хозяйского инвентаря, рабочего скота, недосмотр за хозяйским добром) должник превращался в полного холопа. Известна категория \textbf{«закупа ролейного»}, который вспахивал хозяйскую пашню, смотрел за скотом, выполнял другие сельскохозяйственные работы, пользуясь инвентарем и скотом хозяина. Закуп мог работать, пользуясь хозяйским инвентарем и скотом на своем поле.

\textbf{Закуп находился под защитой публичного права}. Он мог искать судебной защиты, жаловаться на господина, не мог быть продан в холопы. Если не было других свидетелей, он мог быть свидетелем в суде. Штраф за побои для закупа назначался такой же, как для свободного. Но господин имел право его побить «за дело». Всякая отлучка считалась бегством и сурово наказывалась. Бежавший закуп превращался в холопа. \textbf{Холопы и закупы составляли главную рабочую силу в обширных хозяйствах господствующего класса}.

\hfill

Правовые документы Древней Руси регулировали проблему наследства. \textbf{Семейным имуществом распоряжался мужчина, глава семьи}. \textbf{Право наследования имели сыновья}. Жена в случае смерти мужа получала свое приданое. Отец и братья обязаны были выдать своих дочерей и сестер замуж, обеспечив их приданым.

\subsection{Экономический строй Киевской Руси}

\begin{bclogo}[logo=\bcinfo, couleurBarre=orange, noborder=true, couleur=white]{Информация о материале}
    Источник справочного материала по данной теме: репозиторий \href{https://github.com/BFI-2202/history_notes}{BFI-2202/history\_notes}, автор репозитория — \textbf{Лисид Лаконский}
\end{bclogo}

Для экономики Древнерусского государства периода 9-12 веков характерен \textbf{ранний феодализм}.
Зарождалась сама основа взаимоотношений между государством, сельским хозяйством и феодалами. \textbf{Ядром русской экономики в то время считалось именно сельское хозяйство, которое занимало главенствующее положение}.

Данный период характеризуется \textbf{развитостью товарного хозяйства}, поскольку производилось почти все необходимое. \textbf{Ремесленное дело} развивалось быстрыми темпами, а \textbf{центрами становились города}, между тем и в селах развивались отдельные отрасли. Важнейшую роль играла \textbf{черная металлургия}, поскольку в Древней Руси имелись богатые болотные руды. Всевозможными способами обрабатывалось железо, из него изготавливались различные изделия для хозяйственных нужд, военного дела и так далее.

\hfill

Торговля в Древней Руси играла огромное значение, особенно внешняя. \textbf{Внешняя торговля была довольно сильно развита, являлась важной составляющей экономики древнерусских княжеств}. Из Руси \textbf{вывозились на продажу пушнина, воск, мёд, смола, лён и льняные ткани, серебряные вещи, пряслица из розового шифера, оружие, замки, резная кость и прочее}. А предметами ввоза были предметы роскоши, фрукты, пряности, краски и прочее.

\hfill

Формой налогов в Древней Руси выступала \textbf{дань}, которую выплачивали подвластные племена. Чаще всего \textbf{единицей налогообложения выступал «дым», то есть дом, или семейный очаг}. Размер налога традиционно был в \textbf{одну шкурку с дыма}. В некоторых случаях — например, с племени вятичей, — бралось по монете от рала (плуга). \textbf{Формой сбора дани было полюдье}, когда князь с дружиной с ноября по апрель объезжал подданных.

В \textbf{946 году} после подавления восстания древлян княгиня Ольга провела налоговую реформу, \textbf{упорядочив сбор дани}. Она \textbf{отменила полюдье и установила «уроки», то есть размеры дани}, и создала «погосты» — крепости на пути полюдья, в которых жили княжеские управляющие и куда свозилась дань. \textbf{Такая форма сбора дани и сама дань назывались «повоз»}. При уплате налога подданные получали глиняные печати с княжеским знаком, что освобождало их от повторного сбора. Реформа \textbf{содействовала централизации великокняжеской власти и ослаблению власти племенных князей}.

\pagebreak
\section{Вопрос №3. Политический и сословный строй Киевского государства}

\subsection{Политической строй Киевского государства}

\begin{bclogo}[logo=\bcinfo, couleurBarre=orange, noborder=true, couleur=white]{Информация о материале}
    Источник справочного материала по данной теме: Политический и социальный строй Древней Руси // История Отечества. Курс лекций : сайт. – URL: \href{https://siblec.ru/gumanitarnye-nauki/istoriya-otechestva/2-politicheskij-i-sotsialnyj-stroj-drevnej-rusi}{https://siblec.ru/gumanitarnye-nauki/istoriya-otechestva/2-politicheskij-i-sotsialnyj-stroj-drevnej-rusi} (дата обращения: 03.06.2023)
\end{bclogo}

В процессе образования государственности сформировалась и упрочилась власть \textbf{князя}, верховного правителя племенных образований, а затем и всего Древнерусского государства. \textbf{Князь был высшим судьей и руководителем местной власти}. Судебные штрафы шли в княжескую казну.

Князь \textbf{являлся высшим военачальником}. Он организовал оборону страны, сражался во главе своей дружины, заключал мирные договоры. \textbf{Осуществляя внешнюю политику государства, он являлся главным организатором внешней торговли}, которая стала важнейшей отраслью экономики Древней Руси: князь заключал торговые договоры с Византией и заботился об их выполнении, применяя, если нужно, вооруженную силу. Княжеская дружина охраняла товары на торговых путях.

\hfill

Неотъемлемой частью княжеской власти была его \textbf{дружина}, с которой он был неразлучен. Это его ближайшие сподвижники в ратных подвигах, товарищи и советники. Дружина была, как правило, сравнительно невелика и состояла из 700—800 храбрых, обученных и преданных князю воинов, которых он сам подбирал.

\textbf{Князь был обязан постоянно советоваться со своей дружиной, прислушиваться к ее мнению}. Святослав, например, отказался принять христианство только потому, что дружина была против. Игорь погиб из-за того, что, послушавшись совета своей дружины, пошел вторично собирать дань. Владимир Святославович победил своего брата Ярополка благодаря поддержке его Блудом, дружинником Ярополка.

Князь заботился о своей дружине, раздавая ей серебро и злато, лучшую пищу и питье. Дружина делила с князем лавры победы и горечь поражения. После битвы победивший князь расправлялся с дружинниками своего побежденного противника.

\hfill

Княжеская \textbf{дружина не была однородна}. Уже в Х в. она \textbf{делилась на старшую дружину — бояре, или боляре} (от слова “большие”), и \textbf{младшую — молодь, отроки, гридь}. Ярослав, кроме того, упоминает \textbf{нарочитых, или добрых, людей, не входящих в дружину}. Иногда князь собирал ополчение из горожан и свободного сельского населения, которое участвовало в битвах. Так, новгородское ополчение помогло Ярославу Мудрому в борьбе с его братом Святополком за Киевский стол. Постепенно бояре-дружинники перестают жить с князем “на един хлебе”, а \textbf{оседают на земле, владеют селами, превращаются в землевладельцев-феодалов}.

\hfill

\textbf{Издревле среди славянских племен существовало вече — народное собрание}. С образованием государства и усилением власти князя \textbf{роль веча уменьшалась}. В более полном виде оно сохранилось в Новгороде и Пскове. Но и в других княжествах власть князя не была беспредельной: он \textbf{должен был считаться с волеизъявлением народа}. Население Киева и других крупных городов, например, вольно было принять или не принять князя. Вече решало вопросы войны и мира, и прежде всего тогда, когда дружине князя нужно было выставить в помощь ополчение. Иногда народ сам выступал инициатором войны, а в других случаях требовал от князя заключения мира. В междоусобных войнах вече чаще всего выступало как инициатор примирения, но не всегда. Известны случаи, когда жители Киева требовали от князя убрать неугодного им тиуна (управляющего делами князя), который притеснял их поборами и самоуправством. Князь вынужден был выполнить это требование и даже обещал впредь советоваться с ними при назначении тиуна.

\hfill

Как видим, полномочия веча были довольно широки, никакой закон не ограничивал и не определял их. Собрания начинались стихийно, чаще всего при возникновении какой-либо острой ситуации. Иногда ополчение собирало вече во время похода. \textbf{Нельзя также преувеличивать роль веча}. Когда власть князя была крепка, он действовал самостоятельно, и совет веча был для него необязателен. Но \textbf{во многих случаях он должен был считаться с волей народа}.

\pagebreak
\subsection{Сословный строй Киевского государства}

\begin{bclogo}[logo=\bcinfo, couleurBarre=orange, noborder=true, couleur=white]{Примечание}
    Данный материал доступен в справочном материале по предыдущему вопросу
\end{bclogo}

\end{document}